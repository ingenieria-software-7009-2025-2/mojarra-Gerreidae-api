El objetivo de este documento es definir claramente los requerimientos funcionales y no funcionales necesarios para el desarrollo de un sistema destinado a la gestión de incidentes urbanos. Durante el desarrollo del proyecto se busca llevar a cabo un método incremental que permita entregar en cada fase resultados parciales que puedan ser validados por el cliente.

\subsection{Propósito del sistema.}

Se busca desarrollar una página web para el registro y gestión de incidentes urbanos por medio de la participación activa de los ciudadanos. Esto con el fin de a largo plazo hacer uso de los datos recopilados para la optimización de la resolución y seguimiento brindado por las autoridades gubernamentales.

\subsection{Alcance del sistema.}

Dado que el proyecto busca fomentar la participación de la ciudadanía, la funcionalidad prioritaria será permitir a los usuarios (ciudadanos de la CDMX) registrarse en la plataforma web, lo que les habilitará para crear, actualizar y visualizar incidentes relacionados con la vía pública. Los registros de incidentes incluirán información esencial como la ubicación y una descripción del problema, permitiendo un seguimiento adecuado.

\subsection{Definiciones y abreviaciones.}
\begin{itemize}
    \item Salting: Técnica criptográfica que consiste en añadir un valor aleatorio único (llamado salt) a cada contraseña antes de calcular su hash. 
    \item Hashing: Forma de cifrar las contraseñas a través de una función hash criptográfica. Esto para tener más seguridad.
    \item Estado de un accidente: Puede ser "Reportado", "En proceso" o "Resuelto". Se refiere a etiquetas asociadas a los accidentes registrados.
    \item MojarraDrive: Nombre de la página web a implementar.
\end{itemize}

