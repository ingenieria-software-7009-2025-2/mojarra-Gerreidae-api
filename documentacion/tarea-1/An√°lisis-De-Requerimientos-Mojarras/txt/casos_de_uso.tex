%Me rifo esto 

\textbf{\large CU1: Registrar un Incidente} \\

\textbf{Actores:} Usuario \\

\textbf{Flujo Principal:}
\begin{enumerate}
    \item El usuario inicia sesión en la aplicación web \texttt{MojarraDrive}.
    \item Selecciona la opción "Reportar incidente".
    \item Ingresa la ubicación del incidente en el mapa interactivo.
    \item Adjunta una o más fotografías del incidente.
    \item Clasifica el incidente según sus características
    \item Escribe una breve descripción del incidente.
    \item Confirma el registro del incidente.
    \item El sistema guarda y verifica la información.
    \item Una vez se haya confirmado la validez del incidente se muestra en el mapa de la aplicación.
\end{enumerate}

\textbf{Flujo Alternativo:}
\begin{itemize}
    \item Si se detecta algún error en la información o falta información obligatoria, el sistema notificará al usuario y solicitará completar o corregir los campos faltantes. De no completarlo correctamente, el incidente no se mostrará en el mapa. 
\end{itemize}

\vspace{0.5cm}

\textbf{\large CU2: Actualizar Estado de un Incidente} \\

\textbf{Actores:} Usuario \\

\textbf{Flujo Principal:}
\begin{enumerate}
    \item El usuario inicia sesión en la aplicación web \texttt{MojarraDrive}.
    \item Busca un incidente en el mapa o en la lista de reportes.
    \item Selecciona el incidente y elige la opción "Actualizar estado".
    \item Agrega el nuevo estado del incidente.
    \item Adjunta evidencia fotográfica del nuevo estado del incidente.
    \item Confirma la actualización del estado.
    \item El sistema cambia el estado del incidente y almacena el historial de modificaciones.
\end{enumerate}

\textbf{Flujo Alternativo:}
\begin{itemize}
    \item Si el usuario no adjunta pruebas fotográficas, el sistema rechazará la actualización del estado.
\end{itemize}

\vspace{0.5cm}

\textbf{\large CU3: Visualizar Incidentes} \\

\textbf{Actores:} Usuario \\

\textbf{Flujo Principal:}
\begin{enumerate}
    \item El usuario accede a la aplicación web \texttt{MojarraDrive}.
    \item Visualiza un mapa interactivo con incidentes reportados.
    \item Aplica filtros por tipo de incidente, estado y lugar.
    \item Selecciona un incidente para ver detalles y fotografías adjuntas.
\end{enumerate}

\vspace{0.5cm}

\textbf{\large CU4: Administración de Incidentes} \\

\textbf{Actores:} Administrador \\

\textbf{Flujo Principal:}
\begin{enumerate}
    \item El administrador inicia sesión en la aplicación web
    \texttt{MojarraDrive}.
    \item Accede al panel de administración.
    \item Filtra incidentes por tipo o estado.
    \item Modifica, elimina o marca incidentes como revisados.
    \item Genera estadísticas sobre incidentes registrados.
\end{enumerate}
