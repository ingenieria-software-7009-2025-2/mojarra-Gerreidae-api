

Este proyecto fue generado utilizando **Angular**, con [Angular CLI](https://github.com/angular/angular-cli) versión 19.2.1.

Para ejecutar el proyecto en un entorno local, sigue estos pasos:

\begin{itemize}
    \item Clonar el repositorio del backend 
    \begin{verbatim}
        git clone https://github.com/ingenieria-software-7009-2025-2/mojarra-Gerreidae-api
    \end{verbatim}
    \item Ejecutar la aplicación MojarraGerreidaeApiApplication.kt ubicada en el repositorio mojarra-gerreidae-api    
    \item Clonar el repositorio del frontend
    \begin{verbatim}
        git clone https://github.com/ingenieria-software-7009-2025-2/mojarra-Gerreidae-frontend.git
        cd mojarra-Gerreidae-frontend
    \end{verbatim}
    \item Instalar dependencias
    \begin{verbatim}
        npm install
    \end{verbatim}
    \item Iniciar el servidor de desarrollo
    \begin{verbatim}
        ng serve
    \end{verbatim}
    \item Abrir la aplicación en el navegador en
    \begin{verbatim}
        http://localhost:4200/
    \end{verbatim}
\end{itemize}  

La aplicación se recargará automáticamente cuando modifiques los archivos fuente.

\textbf{Servidor de desarrollo
}
Para iniciar el servidor de desarrollo, usa:

\begin{verbatim}
    ng serve
\end{verbatim}


\textbf{Generación de código}

Para generar un nuevo componente, usa:

\begin{verbatim}
    ng generate component nombre-del-componente
\end{verbatim}

\textbf{Construcción del proyecto}

Para compilar el proyecto, usa:

\begin{verbatim}
    ng build
\end{verbatim}

Esto generará los archivos compilados en la carpeta `dist/`. De forma predeterminada, la compilación para producción optimiza la aplicación para mejorar el rendimiento.

\textbf{Ejecutar pruebas unitarias}

Para ejecutar las pruebas unitarias con [Karma](https://karma-runner.github.io), usa:

\begin{verbatim}
    ng test
\end{verbatim}

\textbf{Pruebas end-to-end}

Para ejecutar pruebas end-to-end:

\begin{verbatim}
    ng e2e
\end{verbatim}

Angular CLI no incluye un framework de pruebas e2e por defecto, puedes agregar un paquete que lo implemente.

\textbf{Recursos adicionales}

Para más información sobre Angular CLI y su documentación oficial, visita la página de [Angular CLI Overview and Command Reference](https://angular.dev/tools/cli).
